\documentclass{article}
\usepackage[utf8]{inputenc}
\usepackage[T1]{fontenc}
\title{Fast vertex cover with double star graphs}
\date{2018\\ May}
\author{Samuel Chase\\ Based on \textit{Covering by trees of small diameter} by Lov\'asz}
\begin{document}
	\maketitle
	\section{Introduction}
	In his paper \textit{Covering by trees of small diameter}, Lov\'asz discusses double star trees. Double star trees are trees of diameter less than or equal to 3. A double star can be formed by taking any two star graphs, and drawing an edge between the central nodes of the star, as seen in Figure 1
	\\\\\\
	Double stars also have the interesting property of being able to cover grap hs incredibly quickly. In fact, a graph of n vertices be covered by a greedy algorithm taking at most $2n/3$ steps. All that is required to cover a graph with double stars is to find a maximal matching of edges, and then draw double stars centered around those edges,where the two vertices connected to the edge form the individual single stars. Lov\'asz has proved this to be true, and that $2n/3$ double stars are required. Discussed in this paper will be the key determinants in a maximal double-star cover, as well as examples of such covers on simple graphs.
	\section{The algorithm}
\end{document}