\documentclass{article}
\usepackage[utf8]{inputenc}
\usepackage[english]{babel}
\usepackage{amsthm}

\newtheorem{theorem}{Theorem}
 
\begin{document}
\section{Spanning trees in connected graphs with per-vertex degree constraints}
Assume that we wish to create a spanning tree for a graph. Correct algorithms for this problem proven by Kruskal and other
mathematicians, however in this case we wish to consider an additional constraint. Assume that in the graph every vertex has some degree $n$,
where for a spanning tree to be correct every vertex in the spanning tree must have a degree exactly equal to that degree $n$
 
\begin{theorem}
Given a strongly connected graph $G(E,V)$ with total vertex degree of $2(n-1)$ algorithm 2 will return a spanning tree
\end{theorem}
\begin{proof}
%NOTE! CAN I JUST SAY THIS IS A MODIFIED PRIM'S ALGORITHM?
Assume for $G$, our algorithm is unable to complete a spanning tree. Because our algorithm draws $v-1$ edges and only draws edges to vertices not yet touched, our graph will always attempt to touch every vertex, not create a cycle, and have the correct amount of edges. Therefore for our graph algorithm to fail a degree constraint must be violated, meaning $T$ must have a total degree of zero whilst $T$ does not contain every edge in $G$ preventing more edges from being drawn.
\\
\\
For $G$, $T$ has an initial value equal to degree of the two largest vertices minus 2. All other vertices have a degree equal to the degree of $G \setminus T$. When attempting to pick the next vertex, there are 2 options:\par
    \setlength\parindent{12pt} 
    Case 1) The next vertex selected has a degree = 1 \par
        \setlength\parindent{24pt}\hangindent=24pt 
        In this case, if the largest vertex remaining has a degree of 1 then all vertices not in $T$ must
        have a degree of 1. Additionally, $T$ must have a degree of $2(n-1)-n$ (or $n-2$) Therefore, as there are $n-2$ vertices not in $T$ and connecting
        a vertex to $T$ reduces the total degree of $T$ by 1 it is absurd to state that $T$ will ever be zero.
        \par
        
    \setlength\parindent{12pt}
    Case 2) The next vertex selected has a degree $\geq$ 1\par
        \setlength\parindent{24pt}\hangindent=24pt 
        Adding this vertex to $T$ will increase $T$ by some value greater than 1, and then decrease $T$ by two.
        Therefore the change in $T$ cannot be less than zero, and thus $T$ will never be zero
        \par

    \setlength\parindent{12pt}
    Case 3) No new vertex can be selected\par
        \setlength\parindent{24pt}\hangindent=24pt 
        If no new vertex can be selected this means every vertex in $G$ in in $T$
        \par
        
    \setlength\parindent{0pt}\hangindent=0pt
    Therefore it is absurd that $T$ will ever have a total degree of zero when there are vertices in $G$ not in $T$

\end{proof}
\clearpage 
\begin{theorem}
Given a strongly connected graph $G(E,V)$ with total vertex degree of $2(n-1)$ if algorithm 2 returns a spanning tree it will be correct
\end{theorem}

\begin{proof}
Assume our algorithm returns an incorrect spanning tree for $G$. Our algorithm must necessarily cover every edge, as all the edges will have connections equal to their degree, and no degree can be zero. Additionally our spanning tree will feature $n-1$ edges, where n is the number of vertices, due to the total vertex degree constraint of $2(n-1)$. Therefore, as every vertex is touched, and the total number of edges is equal to $n-1$, the returned tree must be a spanning tree.
\end{proof}

\end{document}