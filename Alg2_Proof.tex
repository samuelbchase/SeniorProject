\documentclass{article}
\usepackage[utf8]{inputenc}
\usepackage[english]{babel}
\usepackage{amsthm}

\newtheorem{theorem}{Theorem}
 
\begin{document}
\section{Spanning trees in connected graphs with per-vertex degree constraints}
Assume that we wish to create a spanning tree for a graph. Correct algorithms for this problem proven by Kruskal and other
mathematicians, however in this case we wish to consider an additional constraint. Assume that in the graph every vertex has some degree constraint $d$,
where for a spanning tree to be correct every vertex in the spanning tree must have a degree exactly equal to that degree constraint $d$
 
\begin{theorem}
Given a strongly connected graph $G(E,V)$ with total vertex degree constraint of $2(|V|-1)$ algorithm 2 will return a correct spanning tree
\end{theorem}
\begin{proof}
Imagine an initial tree $T$, at the completion of algorithm step 3. $G \setminus T$ will have a total degree constraint of $2(|V|-1)-2$, accounting for the single edge in T. Therefore, because it has a total degree constraint of $2(|V|-2)$ and needs to connect $|V|-2$ vertices to $T$. Thus the spanning tree will be able to be completed as the number of possible edges is equal to twice the number of vertices, and $T$ will either have a total degree constraint $\geq$ 1 or be $G$. 
\\
\\
Now, assume $T$ has grown to size k vertices correctly, where $T$ has a total degree constraint large enough to complete the spanning of the graph
\\\\
$T$ will now select the next untouched vertex of maximum degree. There are 3 cases for this vertex:
\\

    \setlength\parindent{12pt} 
    Case 1) The next vertex selected has a degree = 1 \par
        \setlength\parindent{24pt}\hangindent=24pt 
        In this case, if the largest vertex remaining has a degree constraint of 1 then all vertices not in $T$, $U$, must
        have a degree constraint of 1. Thus $T$ must have a total remaining vertex constraint of $|V \setminus U|$, which is equal to  $|U|$. Therefore as there are $|U|$ vertices not in $T$ and $T$ has a total degree constraint of $|U|$ we know the degree constraint will not be violated, as each connection from $T$ to $U$ lowers the total degree constraint by one.
        \par
        
    \setlength\parindent{12pt}
    Case 2) The next vertex selected has a degree $\geq$ 1\par
        \setlength\parindent{24pt}\hangindent=24pt 
        Adding this vertex to $T$ will increase $T$'s total degree constraint by some value $\geq$ 2, and then decrease $T$'s total degree constraint by 2.
        Therefore the change in $T$ cannot be less than zero, and thus $T$ will not be decreased and therefore $T$ will be able to complete the spanning tree.
        \par

    \setlength\parindent{12pt}
    Case 3) No new vertex can be selected\par
        \setlength\parindent{24pt}\hangindent=24pt 
        If no new vertex can be selected this means every vertex in $G$ is in $T$
        \par
        
    \setlength\parindent{0pt}\hangindent=0pt
    Therefore it is clear that $T$ will not violate degree constraints at every step of the algorithm,

\end{proof}
\end{document}